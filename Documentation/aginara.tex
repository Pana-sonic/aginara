\documentclass[a4paper,10pt]{report}

\usepackage{xltxtra}
\usepackage{xgreek}
\usepackage{gfsneohellenic}
\usepackage{hyperref}

\setromanfont[Mapping=tex-text]{GFS Neohellenic}
% \setsansfont[Mapping=tex-text]{DejaVu Sans}
% \setmonofont[Mapping=tex-text]{DejaVu Sans Mono}

\title{Aginara\\
\vspace{0.5cm}
\centering
\includegraphics[scale=0.6]{artichoke.jpg}
\vspace{0.5cm}
}
\author{Γιάννης Κατάκης \\Γιώργος Παναγόπουλος\\ Νεκταρία Ρέκκα\\Ανδρέας Γρίβας}
\date{\today}

\begin{document}
\maketitle

\newpage
\section{Idea}
Η αγκινάρα είναι μια web εφαρμογή που έχει σκοπό να συμβουλεύει τον χρήστη οπτικά για την 
βέλτιστη επιλογή καταστήματος για μια λίστα
προιόντων που επιθυμεί να αποκτήσει. Η εφαρμογή μπορεί να συμβουλέψει τον χρήστη από άποψη: 
\begin{itemize}
 \item Τιμής
 \item Πληρότητας καλαθιού
\end{itemize}

Η Εφαρμογή μας χρησιμοποιεί το \href{http://www.data.gov.gr/dataset/17/}{dataset} με τις τιμές 
προιόντων καταναλωτών.

\section{WebService API}

Τα webservice που στήσαμε είναι Rest και επιστρέφουν json. Ακολουθεί περιγραφή των 
διαθέσιμων μεθόδων.
\subsection{Functions}

\begin{verbatim}
HashMap <String marketName,Double distanceFromUser> superMAPket(String ip); 
ArrayList  <MarketResults> smartList(ArrayList <String> marketNames); 
TODO : 
HashMap <String marketName,Double distanceFromUser> superMAPket(String ip,String address); 

// to call threw ajax

HashMap <String productName,int id> findProduct(String substring);

Double getMinPrice(Product p); //looks for one product everywhere
Double getMinPrice(Product p,String location);//looks for one product in a certain location
Double getMinPrice(Product p,String marketNames); //looks for one product 
//everywhere at marketName markets
Double getMinPrice(Product p,String marketName, String location);
Double getMaxPrice(Product p); //looks for one product everywhere
Double getMaxPrice(Product p,String location); looks for one product in a certain location
Double getMaxPrice(Product p,String marketNames); //looks for one product everywhere 
//at marketName markets
Double getMaxPrice(Product p,String marketName, String location);
Double getAveragePrice(Product p); //looks for one product everywhere
Double getAveragePrice(Product p,String location); looks for one product in a certain location
Double getAveragePrice(Product p,String marketNames); //looks for one product everywhere 
//at marketName markets
Double getAveragePrice(Product p,String marketName, String location);

\end{verbatim}

\subsection{Classes}

\begin{verbatim}

public Market{
    private String marketName;
    private String address;
    private double latitude;
    private double longitude;
}

public Product{
    private int id;
    private String name;
    private double min,max,average;
}

public MarketResults{
    private Market market;
    private ArrayList <String> products;
    private double distance;
    private int percentage;
    private int actualSum; //sum of prices of products
    private int projectedSum; //expected sum of products
}

public PriceLocation{
    private String marketName;
    private double price;
}

\end{verbatim}

\end{document}
