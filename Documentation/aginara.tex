\documentclass[a4paper,10pt]{report}

\usepackage{xltxtra}
\usepackage{xgreek}
\usepackage{gfsneohellenic}
\usepackage{hyperref}
\usepackage[usenames,dvipsnames,svgnames,table]{xcolor}
\usepackage{graphicx}

\setromanfont[Mapping=tex-text]{GFS Neohellenic}
% \setsansfont[Mapping=tex-text]{DejaVu Sans}
% \setmonofont[Mapping=tex-text]{DejaVu Sans Mono}

\renewcommand{\thesection}{\Roman{section}} % Αρίθμηση ενοτήτων

\hypersetup{ % Ρύθμιση των χρωμάτων συνδέσμων
    colorlinks, linkcolor={cyan},
    citecolor={purple}, urlcolor={orange}
}

\title{
\vspace{-1cm}
\centering
\includegraphics[scale=0.8]{aginara-logo3.png}
\vspace{1cm}
\includegraphics[scale=0.6]{artichoke.jpg}
\vspace{0.5cm}
}
\author{Γιάννης Κατάκης \\Γιώργος Παναγόπουλος\\ Νεκταρία Ρέκκα\\Ανδρέας Γρίβας}
\date{\today}

\begin{document}
\maketitle
\tableofcontents

\newpage
\section{Ιδέα}
Η αγκινάρα είναι μια web εφαρμογή που έχει σκοπό να συμβουλεύει τον χρήστη οπτικά για την 
βέλτιστη επιλογή καταστήματος για μια λίστα
προιόντων που επιθυμεί να αποκτήσει. Η εφαρμογή μπορεί να συμβουλέψει τον χρήστη από άποψη: 
\begin{itemize}
 \item Τιμής
 \item Πληρότητας καλαθιού
\end{itemize}
Η Εφαρμογή μας χρησιμοποιεί το \href{http://www.data.gov.gr/dataset/17/}{dataset} με τις τιμές 
προιόντων καταναλωτών.

\section{Παράδειγμα Εκτέλεσης}

\begin{enumerate}
 \item Ο χρήστης εισάγει την διεύθυνση του
 \item Με βοήθεια από το autocomplete συμπληρώνει την λίστα προιόντων που τον ενδιαφέρουν
 \item Πατάει στο κουμπί search
 \item Ανάλογα με την επιλογή που έχει ενεργή επιστρέφονται τα καταστήματα που βρίσκονται σε ακτίνα
 4km και γίνεται οπτικοποίηση των βέλτιστων αποτελεσμάτων
 \item Με click πάνω στο κατάστημα εμφανίζονται περισσότερες πληροφορίες σχετικά με την επιλογή
\end{enumerate}

\section{Δεδομένα}
\subsection{Βάση}
Χρησιμοποιώντας το \href{http://www.data.gov.gr/dataset/17/}{dataset} από το data.gov.gr 
φτιάξαμε μια βάση δεδομένων με τα παραπάνω δεδομένα. Ωστόσο πέρα από αυτό χρησιμοποιήσαμε
api googlemaps και προσθέσαμε έτσι στο dataset και την τοποθεσία με συντεταγμένες των 
καταστημάτων για όσα καταστήματα είχαν σωστή - ολοκληρωμένη διεύθυνση. 
Τα δεδομένα που έχουμε στην βάση μας για όσα καταστήματα είχαν σωστή 
ολοκληρωμένη διεύθυνση, είναι: \\
Για τα καταστήματα :
\begin{itemize}
 \item Όνομα καταστήματος με διεύθυνση
 \item Περιοχή ( Νομό / Περιφέρεια / Δήμος )
 \item Διεύθυνση 
 \item Δήμος 
 \item Δήμος σε Ονομαστική πτώση
 \item Γεωχωρικές Συντεταγμένες καταστήματος
\end{itemize}
Για τα προιόντα :
\begin{itemize}
 \item Όνομα προιόντος
 \item Ημερομηνία εγγραφής προιόντος 
 \item Τιμή προιόντος
\end{itemize}


\subsection{Επαναχρησιμοποίηση}
Στο \href{https://github.com/Pana-sonic/aginara/tree/master/Database}{git} υπάρχουν τα script 
για την δημιουργία της βάσης σύμφωνα με όσα dataset υπάρχουν στον φάκελο Datasets. Συνεπώς αρκεί
κανείς να πάρει τα csv αρχεία που έχουν τα δεδομένα και να τα βάλουν στον φάκελο Datasets και στην
συνέχεια να τρέξουν το reset.sh για να στηθεί η βάση και να έχουν πρόσβαση σε όλη την πρόσθετη
λειτουργικότητα. Επίσης θα μπορούσε κανείς μετά να φτιάξει ένα webservice για να δώσει τις 
συντεταγμένες των καταστημάτων προς τα έξω εφόσον αυτό είναι επιθυμητό.

\section{WebService API}

Τα webservice που στήσαμε και είναι υλοποιημένα σε Java, δίνουν Rest api και επιστρέφουν json.
Ακολουθεί περιγραφή των διαθέσιμων μεθόδων.
\subsection{Headers}
\begin{verbatim}
public String marketCarts(@FormParam("latitude") double latitude,
			   @FormParam("longtitude") double longtitude,
			   @FormParam("products") String products)
			   
public String minList(@FormParam("latitude") double latitude,
			   @FormParam("longtitude") double longtitude,
			   @FormParam("products") String products)
\end{verbatim}

\subsection{Arguments}
\begin{itemize}
 \item marketCarts
 \subitem double latitude
 \subitem double longitude
 \subitem String products
 \subsubitem product\_name:amount
 \item minList
 \subitem double latitude
 \subitem double longitude
 \subitem String products
 \subsubitem product\_name:amount
\end{itemize}
\subsection{Παραδείγματα Κλήσεων}
\begin{itemize}
 \item marketCarts(37.9381,23.6394,'COCA COLA  PET ΦΙΑΛΗ 1.5Lt:1,ΜΠΙΣΚΟΤΑ ΠΤΙ ΜΠΕΡ ΠΑΠΑΔΟΠΟΥΛΟΥ 225gr:2')
 \item minList(37.9381,23.6394,'COCA COLA  PET ΦΙΑΛΗ 1.5Lt:1,ΜΠΙΣΚΟΤΑ ΠΤΙ ΜΠΕΡ ΠΑΠΑΔΟΠΟΥΛΟΥ 225gr:2')
\end{itemize}

\subsubsection{Παραδείγματα Επιστροφής}
\begin{verbatim}
 [
 ["ΣΚΛΑΒΕΝΙΤΗΣ ΠΕΙΡΑΙΩΣ (Ακτή Μουτσοπούλου 48):3.27:3.27:0.624724806994612:false:37.9364:23.6457:100.0",
 "ΜΠΙΣΚΟΤΑ ΠΤΙ ΜΠΕΡ ΠΑΠΑΔΟΠΟΥΛΟΥ 225gr:1.7",
 "COCA COLA  PET ΦΙΑΛΗ 1.5Lt:1.57"],
 ["ΒΕΡΟΠΟΥΛΟΣ ΝΙΚΑΙΑΣ (Αιτωλικού 198):3.41:3.41:2.82474633429836:true:37.9635:23.6399:100.0",
 "ΜΠΙΣΚΟΤΑ ΠΤΙ ΜΠΕΡ ΠΑΠΑΔΟΠΟΥΛΟΥ 225gr:1.84",
 "COCA COLA  PET ΦΙΑΛΗ 1.5Lt:1.57"],
 ["ΣΚΛΑΒΕΝΙΤΗΣ ΠΕΙΡΑΙΩΣ (Σμολένσκυ 18):3.27:3.27:2.972307259341269:false:37.9469:23.6691:100.0",
 "ΜΠΙΣΚΟΤΑ ΠΤΙ ΜΠΕΡ ΠΑΠΑΔΟΠΟΥΛΟΥ 225gr:1.7",
 "COCA COLA  PET ΦΙΑΛΗ 1.5Lt:1.57"],
 ["CARREFOUR ΠΕΙΡΑΙΩΣ (ΜΠΟΥΜΠΟΥΛΙΝΑΣ 58 ΠΑΣΑΛΙΜΑΝΙ):3.29:3.29:0.8344882394028896:false:37.9398:23.648:100.0",
 "ΜΠΙΣΚΟΤΑ ΠΤΙ ΜΠΕΡ ΠΑΠΑΔΟΠΟΥΛΟΥ 225gr:1.7",
 "COCA COLA  PET ΦΙΑΛΗ 1.5Lt:1.59"],
 ["ΣΚΛΑΒΕΝΙΤΗΣ ΠΕΙΡΑΙΩΣ (Λ. Αθηνών-Πειραιώς 87):3.27:3.27:2.542952779674336:false:37.9509:23.6617:100.0",
 "ΜΠΙΣΚΟΤΑ ΠΤΙ ΜΠΕΡ ΠΑΠΑΔΟΠΟΥΛΟΥ 225gr:1.7",
 "COCA COLA  PET ΦΙΑΛΗ 1.5Lt:1.57"],
 ["CARREFOUR MARINOPOULOS ΠΕΙΡΑΙΩΣ (ΑΙΓΑΛΕΩ 26):3.3000002:3.3000002:1.7140700756639535:false:37.9535:23.6386:100.0",
 "ΜΠΙΣΚΟΤΑ ΠΤΙ ΜΠΕΡ ΠΑΠΑΔΟΠΟΥΛΟΥ 225gr:1.7",
 "COCA COLA  PET ΦΙΑΛΗ 1.5Lt:1.6"],
 ["CARREFOUR MARINOPOULOS ΠΕΙΡΑΙΩΣ (ΜΑΚΡΑΣ ΣΤΟΑΣ 3):3.3000002:3.3000002:0.9111450117563706:false:37.945:23.6446:100.0",
 "ΜΠΙΣΚΟΤΑ ΠΤΙ ΜΠΕΡ ΠΑΠΑΔΟΠΟΥΛΟΥ 225gr:1.7",
 "COCA COLA  PET ΦΙΑΛΗ 1.5Lt:1.6"],
 ["CARREFOUR MARINOPOULOS ΠΕΙΡΑΙΩΣ (ΛΑΜΠΡΑΚΗ ΓΡ. 59):3.3000002:3.3000002:1.3908499557086595:false:37.9438:23.6525:100.0",
 "ΜΠΙΣΚΟΤΑ ΠΤΙ ΜΠΕΡ ΠΑΠΑΔΟΠΟΥΛΟΥ 225gr:1.7",
 "COCA COLA  PET ΦΙΑΛΗ 1.5Lt:1.6"],
 ["ΑΒ ΠΕΙΡΑΙΩΣ (Νικήτα 14 Πειραιάς):3.27:3.27:1.0217615761461913:false:37.9458:23.6453:100.0",
 "ΜΠΙΣΚΟΤΑ ΠΤΙ ΜΠΕΡ ΠΑΠΑΔΟΠΟΥΛΟΥ 225gr:1.7",
 "COCA COLA  PET ΦΙΑΛΗ 1.5Lt:1.57"],
 ["ΑΒ ΠΕΙΡΑΙΩΣ (Α. Παπαναστασίου 21):3.27:3.27:1.6310383794634538:false:37.9369:23.6566:100.0",
 "ΜΠΙΣΚΟΤΑ ΠΤΙ ΜΠΕΡ ΠΑΠΑΔΟΠΟΥΛΟΥ 225gr:1.7",
 "COCA COLA  PET ΦΙΑΛΗ 1.5Lt:1.57"]
 ]
\end{verbatim}
\begin{verbatim}
["CARREFOUR MARINOPOULOS ΠΕΙΡΑΙΩΣ (ΑΙΓΑΛΕΩ 26):ΜΠΙΣΚΟΤΑ ΠΤΙ ΜΠΕΡ ΠΑΠΑΔΟΠΟΥΛΟΥ 225gr:
1.7140700756639535:37.9535:23.6386:0.85",
"ΒΕΡΟΠΟΥΛΟΣ ΝΙΚΑΙΑΣ (Αιτωλικού 198):COCA COLA  PET ΦΙΑΛΗ 1.5Lt:
2.82474633429836:37.9635:23.6399:1.57"]
\end{verbatim}

\section{Μελλοντική Δουλειά}

Στην συνέχεια η ιδέα είναι να εισαχθεί στην εφαρμογή δυνατότητα καταστημάτων 
που δεν υπάρχουν στο dataset να κάνουν εγγραφή στην εφαρμογή - δεν προλάβαμε να 
κάνουμε την υλοποίηση στα πλαίσια του hackathon. Θα δίνεται επιλογή στον χρήστη να 
επιβεβαιώσει πως είναι αποδεκτές οι τιμές για τη λίστα που συμπλήρωσε σε κάποιο 
κατάστημα και πως είναι πρόθυμος να αγοράσει. \\
Τότε θα ενημερώνονται όσα καταστήματα βρίσκονται στην ακτίνα αναζήτησης του χρήστη
και έχουν κάνει εγγραφή στην εφαρμογή και θα τους δίνεται η δυνατότητα να κάνουν
μια καλύτερη προσφορά στον χρήστη για την ίδια λίστα.\\
 

\end{document}
